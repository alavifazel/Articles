% Created 2022-01-16 Sun 22:21
% Intended LaTeX compiler: pdflatex
\documentclass[11pt]{article}
\usepackage[utf8]{inputenc}
\usepackage[T1]{fontenc}
\usepackage{graphicx}
\usepackage{grffile}
\usepackage{longtable}
\usepackage{wrapfig}
\usepackage{rotating}
\usepackage[normalem]{ulem}
\usepackage{amsmath}
\usepackage{textcomp}
\usepackage{amssymb}
\usepackage{capt-of}
\usepackage{hyperref}
\hypersetup{colorlinks=true,linkcolor=blue}
\date{\today}
\title{The idea behind UNIX numeric file permissions}
\hypersetup{
 pdfauthor={Iman Alavi Fazel},
 pdftitle={The idea behind UNIX numeric file permissions},
 pdfkeywords={},
 pdfsubject={},
 pdfcreator={Emacs 27.2 (Org mode 9.4.4)}, 
 pdflang={English}}
\begin{document}

\maketitle
\tableofcontents

Any UNIX-like operating system user is probably familiar with the \textbf{chmod} command.
In short, this command changes the permission of a file (or directory) with the desired permisison level for the \emph{owner}, \emph{group} and \emph{everyone else} in the system.

For instance, the following command:

\begin{quote}
chmod 755 filename.txt
\end{quote}

Changes the permission of \emph{filename.txt} to \textbf{read, write, execute} for the \emph{owner}, and \textbf{read and execute} for \emph{group} and \emph{everyone} respectively.

The way that the number 755 carried this information is that each digit (7 or 5 in this example) is an addition of the following numbers that have special meanings for UNIX:

\begin{itemize}
\item \emph{1} for execution permission.
\item \emph{2} for write permission.
\item \emph{4} for read permission.
\end{itemize}

In our example, 5 was the result of (1 + 4) and the OS concluded that we needed both \textbf{read} and \textbf{execution} permissions.

What I was wondering was how there was a guarantees that a number in this system is only the result of specific numbers?
For example, how can we be sure that 5 is not the result of adding up number other that 1 and 4 in our system?
(Of course, since our choices are limited in this senario we could simply list out all the items that the basis numbers 1, 2 and 4 make and check them all out. But I was looking for a more generalized logic)

A short pondering about this question made me realized that the numbers \emph{1}, \emph{2} and \emph{4}, which are used as basis numbers, are all in base 2!
Hence, the number 5 in binary is \emph{101}, which the OS can clearly conclude that we intended to have both read and execution options for our file.

Can we extend this idea to carry information that aren't consist of \textbf{on-or-off} options like in file permissions?
For this, let's take a look at a case (probably useless and with a poor software design choice) just to have a concret example:

A restaurant application software has 15 types of meals and 6 types of canned soda, and we want to represent the order of a customer (including both the meal and soda type) with a single decimal number.

Let's say we have the following table for \textbf{meals}:

\begin{center}
\begin{tabular}{ll}
ID & Name\\
. & .\\
5 & Chicken Wings\\
. & .\\
\end{tabular}
\end{center}

And the following for \textbf{soda types}:

\begin{center}
\begin{tabular}{ll}
ID & Name\\
. & .\\
2 & Coca-Cola Classic\\
. & .\\
\end{tabular}
\end{center}

Therefore, the number \(5 + (16*2) = 37\) (in decimal) can represent an order of Chicken Wings with Coca-Cola Classics.
When this number is converted to base 16, it yields 25, where each digit (2 and 5) represent an ID from their example tables.

Although this example wasn't practical, it was examplify the generalized the idea behind the concept of file permissions in UNIX. 
\end{document}
